\documentsclass{article}
\begin{document}
•	Team name, team onid, project name: Brice and Leif, Brice ONID (932300054)
•	Problem being solved: Driving and texting (distracted driving)
•	Evidence of problem: According to the FCC, 1161 people daily are injured in distraction related crashes, and 8 are killed.  ODOT also reported that between 2011 and 2015, there were 9951 crashes in Oregon.
•	Tell a brief story of the problem: Amanda Clark, a teen residing in Oakdale, California, was broadsided when she ran a stop sign at an intersection as a result of distracted driving.  She lived, and promised to never text and drive.  One year and a day later, she lost control of her car and crashed off the highway while texting her roommate.  This is only one of the many tragic stories created by distracted driving, but this one in particular shows how much an app such as ours could save lives and help prevent these events from happening.
•	What the app will do to fix problem: Disable the temptation to text with a do not disturb mode while the car has approached high speeds. 
•	High level approach: The app will use google maps along with the gyroscope to tell whether the car is in motion to put the phone into a do not disturb mode.
•	Key differences of this approach to others:  Sleek, fast, lightweight, minimalistic power consumption
•	Features that make this effective or useful: Automatic detection, smooth feeling UI
•	Limitations, if any: Google maps, Android permissions for background apps
•	Resources needed to complete the project: Google Maps, gyroscope, Java or C++
•	Most serious challenges and how to fix it:  Provide an app that is unique among a platform that has been used many times


•	Settings page 
o	Features we can turn off and on – 
	Pleasant sound when activated or deactivated
	Run on device startup
	Parent control
•	Lock settings behind a pass code
•	Don’t allow uninstall without passcode or send notification to parental account when uninstalled
•	Chose which features to disable
•	Notifications
o	Driving habits and max speeds
o	Time driving
•	Set the threshold speed when activated







•	Cite Sources:
o	“The Dangers of Distracted Driving.” Federal Communications Commission, Federal Communications Commission, 8 Sept. 2017, www.fcc.gov/consumers/guides/dangers-texting-while-driving.
o	“MenuToggle Secondary Menu  Distracted Driving.” Oregon Department of Transportation : Distracted Driving : Safety : State of Oregon, Oregon Department of Transportation, www.oregon.gov/ODOT/Safety/Pages/Distracted.aspx.

\end{document}